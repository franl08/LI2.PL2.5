1ª Fase do Projeto (Guião 5)\+: Nesta fase, tudo decorreu dentro da normalidade e, portanto, fomos capazes de implementar todas as funcionalidades que o Guião 5 nos propunha. Como é normal ao iniciar-\/se um projeto de maior dimensão, sentimos dificuldade no início da implementação das funções, pois, não estávamos completamente cientes daquilo que queríamos que o nosso código fosse capaz de fazer, nem como o deveria fazer. Além disso, sentimos também alguma dificuldade na utilização do Git\+Hub, visto que esta é a primeira vez de vários elementos do grupo a utilizarem esta plataforma. Será de realçar também que, sentimos também dificuldades a dar print ao tabuleiro, visto que, este dava print de peças que ainda nem tínhamos declarado, logo, causou-\/nos bastante confusão. Por fim, no capítulo das dificuldades, surgiu também o facto de não sabermos muito bem, ao início, como testar aquilo que íamos fazendo, mas, com o evoluir do nosso envolvimento com o projeto conseguimos colmatar, também, essa dificuldade. Concluindo, apesar das dificuldades, esta 1ª fase decorreu, no nosso ponto de vista, de uma boa maneira, sendo que, fomos capazes de cumprir tudo aquilo que o Guião nos propunha.

2ª Fase do Projeto (Guião 6)\+: Mais uma vez, nesta fase, tudo decorreu dentro da normalidade, ou seja, o nosso grupo conseguiu implementar todas as funcionalidades que o Guião 6 propunha. Desta vez, as maiores dificuldades sentidas foram em como se guardava/lia um ficheiro, visto que, não estavamos, ainda, por dentro de tal assunto, o que nos fez, pesquisar e estudar sobre a matéria. Além disso, também achamos um pouco estranho a quantidade de condições que tivemos de colocar para verificar se um jogador poderia continuar a jogar, ou, se estaria bloqueado, pois, não vimos qualquer maneira de compactar o código. Em suma, apesar de terem surgido algumas dificuldades, a 2ª Fase, a nosso ver, correu bem, pois, tudo aquilo que nos foi proposto, nós fomos capazes de implementar.

3ª Fase do Projeto (Guião 7)\+: Nesta fase, apesar de alguns problemas iniciais, tudo decorreu dentro da normalidade, isto é, o nosso grupo foi capaz de implementar todas as funções propostas. As dificuldades surgidas foram, sobretudo, devido à diferente formatação dos ficheiros, pois, ao contrário da equipa docente, ao implementar o comando gr e o ler utilizamos os ficheiros com espaços entre os diversos pontos do tabuleiro. Concluindo, apesar da preocupação por não compreendermos onde estaria o erro, a 3ª fase, na nossa opinião, correu bem. 